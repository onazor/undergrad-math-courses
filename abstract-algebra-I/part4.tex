\subsection{Friday, September 30: Cyclic Groups}

We recall the definition of the order of an element $g$ in group $G$.

\begin{theorem}
    Let $G$ be a group, $a \in G$ and $i \neq j$, with $i, j \in \Z$. If $a^i = a^j$, then $a$ has finite order.
\end{theorem}

\begin{remark}
    Let $G$ be a group, $a \in G$ and $i \neq j$, with $i, j \in \Z$. If $a$ has infinite order, then $a^i \neq a^j$ (that is, the elements $a^k$ where $k \in \Z$ are all distinct).
\end{remark}

\begin{theorem}
    Let $G$ be a group and $a \in G$. Then $\langle a \rangle = \qty{a^n \mid n \in \Z}$ is a subgroup of $G$, and is the smallest subgroup of $G$ that contains $a$.
\end{theorem}

\begin{definition}[Cyclic Subgroup]
    Let $G$ be a group and $a \in G$. The subgroup $\langle a \rangle = \qty{a^n \mid n \in \Z}$ is called the cyclic subgroup of $G$ generated by $a$.
\end{definition}

\begin{remark}
    Let $G$ be a group and $a \in G$. \\
    \begin{enumerate}
        \item $\langle a \rangle = \langle a^{-1} \rangle$
        \item If $G$ is an infinite group, $\langle a \rangle$, need not be infinite
        \item Let $a, b, \in G$. If $a \in \langle b \rangle$, then $\langle a  \rangle \subseteq \langle b \rangle$
        \item The subgroup $\langle a \rangle$ is abelian
    \end{enumerate}
\end{remark}

\begin{definition}[Cyclic Group]
    A group $G$ is called a cyclic group if $\exists a \in G$ such that $G = \langle a \rangle$. The element $a$ is said to generate $G$ or $a$ is a generator of $G$.
\end{definition}

\begin{remark}
    Let $G = \langle a \rangle$ where $a \in G$.
    \begin{itemize}
        \item $G$ is an abelian group
        \item $a$ is of infinite order $\Rightarrow$ distinct powers of $a$ are distinct elements and $G \cong \Z$
        \item $\qty|a| = n \Rightarrow G = \qty{e, a, \ldots , a^{n-1}}, a^i = a^j$ if and only if $i \equiv j \pmod{n}$ and $G \cong \Z_n$
        \item every subgroup of $G$ is cyclic
    \end{itemize}
\end{remark}

\begin{exercise}
    Suppose $G = \qty{e, t, u, v, w, x, y, z}$ is the group defined by the table below.
    \begin{center}
        \begin{tabular}{c|c c c c c c c c}
             $\ast$ & $e$ & $t$ & $u$ & $v$ & $w$ & $x$ & $y$ & $z$  \\
             \hline
             $e$ & $e$ & $t$ & $u$ & $v$ & $w$ & $x$ & $y$ & $z$  \\
             $t$ & $t$ & $e$ & $v$ & $u$ & $y$ & $z$ & $w$ & $x$  \\
             $u$ & $u$ & $v$ & $e$ & $t$ & $z$ & $y$ & $x$ & $w$  \\
             $v$ & $v$ & $u$ & $t$ & $e$ & $x$ & $w$ & $z$ & $y$  \\
             $w$ & $w$ & $y$ & $x$ & $z$ & $t$ & $v$ & $e$ & $u$  \\
             $x$ & $x$ & $z$ & $w$ & $y$ & $u$ & $e$ & $v$ & $t$  \\
             $y$ & $y$ & $w$ & $z$ & $x$ & $e$ & $u$ & $t$ & $v$  \\
             $z$ & $z$ & $x$ & $y$ & $w$ & $v$ & $t$ & $u$ & $e$  \\
        \end{tabular}
    \end{center}
    Find the elements of $\langle x \rangle$ and $\langle y \rangle$
\end{exercise}

\begin{solution}
    We have the following
    \[ \langle x \rangle = \qty{e, x} \]
    and
    \[ \langle y \rangle = \qty{e, y,  t, w} \]
\end{solution}

\begin{exercise}
    Determine whether the following are cyclic groups
    \begin{enumerate}[(a)]
        \item $U\qty(\Z_{10}) = \qty{1, 3, 7, 9}$
        \item $U\qty(\Z_8) = \qty{1, 3, 5, 7}$
    \end{enumerate}
\end{exercise}

\begin{solution} \phantom{blank}
    \begin{enumerate}[(a)] 
        \item We will see what element generates the group. We have
        \[ \langle 1 \rangle = \qty{1} \]
        \[ \langle 3 \rangle = \qty{1, 3, 9, 7}\]
        Since there is a generator. Then it is a cyclic group. Furthermore, this is also isomorphic to $\Z_4$
        \item Similarly, 
                \[ \langle 1 \rangle = \qty{1} \]
        \[ \langle 3 \rangle = \qty{1, 3} \]
        \[ \langle 5 \rangle = \qty{1, 5} \]
        \[ \langle 7 \rangle = \qty{1 ,7} \]
        Since there is no generator, then this is not cyclic group. However, this group is isomorphic to $V_4$ (Klein-4 group)
    \end{enumerate}
\end{solution}

\begin{theorem}
    Let $G$ be a group and $a \in G$. Suppose $a$ has finite order $n$. Then,
    \begin{enumerate}[i.]
        \item $\langle a \rangle = \qty{e, a, a^2, a^3, \ldots, a^{n-1}}$
        \item For $i, j \in \Z, a^i = a^j$ if and only if $i \equiv j \pmod{n}$
    \end{enumerate}
\end{theorem}

\begin{remark}
    Let $G$ be a group and $a \in G$.
    \begin{enumerate}
        \item Let $\langle a \rangle$ be a finite cyclic subgroup of $G$. Then $\qty|a| = \qty| \langle a \rangle |$
        \item Suppose $\qty|a| = n$ for some $n \in \Z^+$, and $k \in \Z$. Then $a^k = e$ if and only if $n \mid k$
        \item If $a$ is of infinite order, then
        \[ \langle a \rangle = \qty{\ldots, a^{-2}, a^{-1}, e, a^1, a^2, \ldots} \]
    \end{enumerate} 
\end{remark}

\begin{theorem}
    Every subgoup of a cyclic group is cyclic.
\end{theorem}

\begin{theorem}
    Let $G = \langle a \rangle$ with $\qty|a| = n, k \in \Z^+$ and $d = \text{gcd}\qty(n, k)$. Then,
    \begin{enumerate}
        \item $\langle a^k \rangle = \langle a^d \rangle$ and
        \item $\qty|a^k| = \frac{n}{d}$
    \end{enumerate}
\end{theorem}

\begin{corollary}
    Let $G = \langle a \rangle $ with $\qty|a| = n$ and $m \in \Z^+$. Then $a^m$ is a generator of $G$ if and only if $\text{gcd} \qty(n, m) = 1$
\end{corollary}

\begin{corollary}
    Let $G = \langle a \rangle$ with $\qty|a| = n$. Then for every positive divisor $d$ of $n, \exists! H \leq G$ with $\qty|H| = d$.
\end{corollary}

\begin{remark}
    The distinct subgroups of $G = \langle a \rangle$ where $\qty|a| = n$ are those subgroups $\langle a^d \rangle$ where $d$ is a positive divisor of $n$.
\end{remark}

\begin{exercise}
    Sketch the subgroup lattice of $\Z_{18}$. Find all generators of each distinct subgroups.
\end{exercise}

\begin{solution}
    Recall from Remark 14 that the distinct subgroups is generated by $\langle a^d \rangle$ where $d$ is a positive divisor of $n$. Note that the positive divisors of 18 are 1, 2, 3, 6, 9, 18. Then, we have the following \\
    Subgroup of order 18: (We'll use Corollary 2)
    \[ \Z_{18} = \langle 1 \rangle = \langle 5 \rangle = \langle 7 \rangle = \langle 11 \rangle = \langle 13 \rangle = \langle 17 \rangle \]
    Subgorup of order 9: (From Corollary 2, $\text{gcd}\qty(9, k) = 1$)
    \[ \bigg\langle \frac{18}{9} \cdot 1 \bigg\rangle = \langle 2 \rangle = \langle 4 \rangle = \langle 8 \rangle = \langle 10 \rangle = \langle 14 \rangle = \langle 16 \rangle = \qty{0, 2, 4, 6, 8, 10, 12, 14, 16}\]
    Subgroup of order 6: ($\text{gcd}\qty(6, k) = 1$)
    \[ \bigg\langle \frac{18}{6} \cdot 1 \bigg\rangle  = \langle 3 \rangle = \langle 15 \rangle = \qty{0, 3, 6, 9, 12, 15}  \]
    Subgroup of order 3: ($\text{gcd}\qty(3, k) = 1$)
    \[ \bigg\langle \frac{18}{3} \cdot 1\bigg\rangle = \langle 6 \rangle = \langle 12 \rangle = \qty{0, 6, 12 }  \]
    Subgorup of order 2: ($\text{gcd}(2, k) = 1$)
    \[  \bigg\langle \frac{18}{2} \cdot 1 \bigg\rangle  = \langle 9\rangle \]
    Subgroup of order 1:
    \[ \langle 0 \rangle = \qty{0} \]
    The lattice diagram is shown below.
    
    \begin{center}
        \tikzset{every picture/.style={line width=0.75pt}} %set default line width to 0.75pt        
        
        \begin{tikzpicture}[x=0.75pt,y=0.75pt,yscale=-1,xscale=1]
        %uncomment if require: \path (0,300); %set diagram left start at 0, and has height of 300
        
        %Shape: Rectangle [id:dp12954802076539673] 
        \draw   (448.76,111.57) -- (285.1,269.34) -- (216.24,197.9) -- (379.9,40.13) -- cycle ;
        %Straight Lines [id:da07955177804162328] 
        \draw    (299,119.47) -- (367.67,189.47) ;
        
        % Text Node
        \draw (362,15.4) node [anchor=north west][inner sep=0.75pt]    {$\mathbb{Z}_{1}{}_{8}$};
        % Text Node
        \draw (270.67,97.4) node [anchor=north west][inner sep=0.75pt]    {$\langle 3\rangle $};
        % Text Node
        \draw (186.67,186.2) node [anchor=north west][inner sep=0.75pt]    {$\langle 9\rangle $};
        % Text Node
        \draw (281.87,271) node [anchor=north west][inner sep=0.75pt]    {$\langle 0\rangle $};
        % Text Node
        \draw (456.27,99.4) node [anchor=north west][inner sep=0.75pt]    {$\langle 2\rangle $};
        % Text Node
        \draw (374.67,189.8) node [anchor=north west][inner sep=0.75pt]    {$\langle 6\rangle $};
        \end{tikzpicture}
    \end{center}
\end{solution}

\begin{exercise}
    If $a$ is an element of a group where $\qty|a^4| = 12$, what are the possibilities for $\qty|a|$?
\end{exercise}

\begin{solution}
    Since $\qty|a^4| = 12$, then we have $\qty(a^4)^{12} = a^{48} = e$. Thus, we have $\qty|a| \mid 48$. So $48 = \qty|a| \cdot k, \exists k \in \Z^+$. Since $\langle a^4 \rangle \leq \langle a \rangle$, $\qty|\langle a^4 \rangle| \mid \langle a \rangle = \qty|a|$, so $12 \mid \qty|a|$ Therefore, $\qty|a| = 12 \cdot m, \exists m \in \Z^+$. This gives us $48 = \qty|a| \cdot k = 12 \cdot mk \Rightarrow mk = 4$ So the values of $m$ could be $1, 2, 4$. Therefore, we have $\qty|a| = 12, 24, 48$.
    \begin{myspace}
        \begin{enumerate}[label=\textbf{Case \arabic*:}]
            \item If $\qty|a| = 12$, then $\qty|a^4| = \frac{12}{\text{gcd} \qty(12, 4)} = \frac{12}{4} = 3$
            \item If $\qty|a| = 24$, then $\qty|a^4| = \frac{24}{\text{gcd} \qty(24, 4)} = \frac{24}{4} = 6$
            \item If $\qty|a| = 48$, then $\qty|a^4| = \frac{48}{\text{gcd} \qty(48, 4)} = \frac{48}{4} = 12$
        \end{enumerate}
    \end{myspace}
    The first two cases are contradictions. Therefore the order of $\qty|a|$ is 48.
\end{solution}

\begin{exercise}
    Let $G = \langle a \rangle$, with $\qty|a| = 72$. Find
    \begin{enumerate}[(a)]
        \item $\qty|a^{188}|$, and
        \item all generators of the subgroup of $G$ of order 12
    \end{enumerate}
\end{exercise}

\begin{solution} \phantom{blank}
    \begin{enumerate}[(a)]
        \item  Note that $a^{188} = a^{2 \cdot 72 + 44} = \qty(a^{72})^2 \cdot a^{44} = a^{44}$
        \[ \qty|a^{188}| = \qty|a^{44}| = \frac{72}{\text{gcd}\qty(72, 44)} = \frac{72}{4} = 18 \]
        \item We need to find $m$ such that $\text{gcd}(m, 12) = 1$ since $12 = \qty|a^{\frac{72}{12}}| = \qty|\qty(a^6)^m|$ where $\text{gcd}(m, 12) = 1$. Therefore, $m = 1, 5, 7, 11$. Thus, the generators of a subgroup of order 12 are $a^6, a^{30}, a^{42}, a^{66}$.
    \end{enumerate}
\end{solution}

\begin{exercise}
    Suppose $a, b$ are elements of a finite group. Prove:
    \begin{itemize}
        \item If $\qty|a| = \qty|a^2|$, then $\qty|a|$ is odd
        \item $\qty|a| = \qty|b^{-1}ab|$
    \end{itemize}
\end{exercise}

\begin{solution} \phantom{blank}
    \begin{itemize}
        \item Note that $\frac{\qty|a|}{\text{gcd}\qty(\qty|a|, 2)} = \qty|a^2| = \qty|a|$ (group is finite). Therefore, $\text{gcd}\qty(\qty|a|, 2) = 1$. This implies that $\qty|a|$ is odd.
        \item Note that $\qty|b^{-1}ab| < \infty$. Suppose $\qty|a| = m$ and $\qty|b^{-1}ab| = n$ where $m, n \in \Z^+$. This implies that
        \begin{align*}
            e = \qty(b^{-1}ab)^n &= \qty(b^{-1}ab)\qty(b^{-1}ab) \ldots \qty(b^{-1}ab) \\
            &= b^{-1}a^nb
        \end{align*}
        Note that $e = b \cdot e \cdot b^{-1}$. Thus, we can use this to generate
        \[ e = b \cdot e \cdot b^{-1} = b \qty(b^{-1} a^n b) b^{-1} = a^n \]
        Therefore, $m \mid n$. \\
        
        Now, $e = a^m \Rightarrow e = b^{-1}eb = b^{-1}\qty(a^m)b = \qty(b^{-1}ab)^m$. This implies that $n \mid m$ where $m, n \in \Z^+$. \\
        
        Since $m \mid n$ and $n \mid m$, we get $m = n$.
    \end{itemize}
\end{solution}

\begin{exercise}
    Prove that if a group $G$ has no proper nontrivial subgroups, then $G$ is a cyclic group.
\end{exercise}

\begin{proof}
    Let $g \in G$ with $g \neq e$. Then $\langle g \rangle \leq G$ and $\langle g \rangle \neq \qty{e}$. Since $G$ has no proper nontrivial subgroup, $\langle g \rangle = G$. Hence, $G$ is cyclic. \qedsymbol
\end{proof}

\begin{exercise}
    Let $G$ be an abelian group. Show that the elements of finite order in $G$ form a subgroup. This subgroup is called the torsion subgroup of $G$.
\end{exercise}

\begin{proof}
    Let $T = \qty{a \in G \mid a^n = e \exists n \in \Z^+}$. We will show that $T \leq G$ using 3-step subgroup test.
    
    \begin{itemize}
        \item   We know that $e^n = e, \forall n \in \Z^+ \Rightarrow e \in T$. 
        \item Let $a, b \in T$. Then $a^{n_1} e \exists n_1 \in \Z^+$ and $b^{n_2} = e \exists n_2 \in \Z^+$. We have,
        \[ (ab)^{n_1n_2} = a^{n_1n_2}b^{n_1n_2} = \qty(a^{n_1})^{n_2} \cdot \qty(b^{n_2})^{n_1} = e^{n_2} \cdot e^{n_1} = e \text{ and } n_1n_2 \in \Z^+\]
        \item Let $a \in T$. Then $a^n = e \exists n \in \Z^+$. Therefore, $\qty(a^{-1})^n = \qty(a^n)^{-1} = e^{-1} = e $
    \end{itemize}

\end{proof}