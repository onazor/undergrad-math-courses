\subsection{Friday, October 21: Direct Product, Subgroups Generated by a Subset, and Finitely Generated Abelian Groups}

\subsubsection{Direct Product}

\begin{definition}
    Let $\langle G_1, \cdot \rangle$ and $\langle G_2, \cdot \rangle$ be groups. The (external) direct product of $G_1$ and $G_2$ is the group $G_1 \times G_2 = \qty{\qty(g_1, g_2) \mid g_1 \in G_1, g_2 \in G_2}$ with binary operation $\ast$ defined by 
    \[ \qty(a_1, a_2) \ast \qty(b_1, b_2) = \qty(a_1 \cdot b_1, a_2 \cdot' b_2) \]
    for every $\qty(a_1, a_2), \qty(b_1, b_2) \in G_1 \times G_2$
\end{definition}

\begin{remark} \phantom{blank} \\
    \begin{enumerate} 
        \item $e_G$ identity element of group $G$, $e_{G_1 \times G_2} = q\qty(e_{G_1}, e_{G_2}), \forall \qty(g_1, g_2) \in G_1 \times G_2, \qty(g_1, g_2)^{-1} = \qty({g_1}^{-1}, \qty{g_2}^{-1})$
        \item In general, if $G_1, G_2, \ldots, G_n$ are groups, then $\Pi_{i = 1}^{n} G_i$ is a group with binary operation $\ast$ defined by 
        \[ \qty(a_1, a_2, \ldots, a_n) \ast \qty(b_1, b_2, \ldots, b_n) = \qty(a_1b_1, a_2b_2, \ldots a_nb_n) \]
        for every $\qty(a_1, a_2, \ldots, a_n), \qty(b_1, b_2, \ldots, b_n) \in \Pi_{i = 1}^{n} G_i$. Moreover, if $G_1, G_2, \ldots, G_n$ are abelian groups, then $\Pi_{i = 1}^{n} G_i$ is also abelian
        \item If $\qty|G_i| < \infty$, for each $i \in \qty{1, 2, \ldots, n}$, then $\qty|\Pi_{i = 1}^{n} G_i| = \Pi_{i = 1}^{n} \qty|G_i| < \infty$
    \end{enumerate}
\end{remark}

\begin{recall}[Least Common Multiple]
    Let $a, b$ be nonzero integers and $m$ be a positive integer. Then $m$ is the least common multiple of $a$ and $b$ if $m$ satisfies the following:
    \vspace{1em}
    \begin{enumerate}
        \item $a \mid m$ and $b \mid m$, that is $m$ is a multiple of both $a$ and $b$;
        \item $\forall c \in \Z$, if $a \mid c$ and $b \mid c$, then $m \mid c$.
    \end{enumerate}
\end{recall}

\begin{theorem}
    Let $G_1, G_2$ be finite groups. If $\qty(a, b) \in G_1 \times G_2$, then $\qty|\qty(a ,b)| = \text{lcm} \qty(\qty|a|, \qty|b|)$
\end{theorem}

\begin{remark}
    In general, if $G_1, G_2,\ldots, G_n$ are finite groups and $\qty(g_1, g_2, \ldots, g_n) \in \Pi_{i = 1}^{n} G_i$, then $\qty|\qty(g_1, g_2, \ldots, g_n)| = \text{lcm}\qty(\qty|g_1|, \qty|g_2|, \ldots, \qty|g_n|)$
\end{remark}

\begin{example}
    How many elements of order 10 are in $\Z_{25} \times \Z_{100}$?
\end{example}

\begin{solution}
    Let $\qty(a, b) \in \Z_{25} \times \Z_{100}$ such that $\qty|\qty(a, b)| = \text{lcm} \qty(\qty|a|, \qty|b|) =10$. Note that the elements of the order of $\Z_{25}$ is $\qty{1, 5, 25}$ and the order of elements of $\Z_{100}$ is $\qty{1, 2, 4, 5, 10, 20, 25, 50, 100}$. 
    
    \begin{myspace}
        \begin{enumerate}[label=\textbf{Case \arabic*:}]
            \item $\qty|a| = 1$ and $\qty|b| = 10$ \\
            So we have $a=0$ and $b \in \qty{10, 30, 70, 90}$. Therefore, there are 4 elements of $\Z_{25} \times \Z_{100}$ with order 10.
            \item $\qty|a| = 5$ and $\qty|b| = 2$ \\
            So we have $a \in \qty{5, 10, 15, 20}$ and $b = 50$. Therefore, there are 4 elements of $\Z_{25} \times \Z_{100}$ with order 10 for this case. 
            \item $\qty|a| = 5$ and $\qty|b| = 10$ \\
            So we have $a \in \qty{5, 10, 15, 20}$ and $b \in \qty{10, 30, 70, 90}$. Therefore, there are 16 elements of $\Z_{25} \times \Z_{100}$ with order 10 for this case.
        \end{enumerate}
        Hence, there are 24 elements of  $\Z_{25} \times \Z_{100}$ with order 10.
    \end{myspace}
\end{solution}

\begin{exercise}
    Find $\qty|\qty(9, -i, \qty(62547)\qty(3612))|$ where $\qty(9, -i, \qty(62547)\qty(3612)) \in {\Z_{13}}^{\ast} \times U_4 \times S_8$
\end{exercise}

\begin{solution}
    For $\qty|9|$, note that $\langle 2 \rangle = \Z_{13}^{\ast}$ and $\langle 3 \rangle = \qty{1, 3, 9} \Rightarrow \qty|\langle 3 \rangle| = 3$. Thus
    \[ \qty|9| = \qty|3^2| =  \frac{3}{\text{gcd} \qty(3, 2)} = 3 \]
    
    For $-i$, note that $\langle -i \rangle = \qty{-i, -1, i, 1}$. So $\qty|-i| = 4$. 
    
    For $\qty(62547) \qty(3612)$, this is equal to $\qty(15476) \qty(23)$. Therefore, $\qty|\qty(15476) \qty(23)| = \text{lcm} \qty(5, 2) = 10$.
    
    Therefore, 
    \begin{align*}
        \qty|\qty(9, -i, \qty(62547)\qty(3612))| &= \text{lcm}\qty(\qty|9|, \qty|-i|, \qty|\qty(62547)\qty(3612)|) \\
        &= \text{lcm} \qty(3, 4, 10) \\
        &= 60
    \end{align*}
\end{solution}

\begin{remark} \phantom{blank}
    \begin{enumerate}
        \item Let $G_1, G_2$ be groups. Then $G_1 \times G_2 \cong G_2 \times G_1$
        \item Suppose that $\Pi_{i = 1}^{n} G_i$ is the direct product of groups $G_i$,
        \begin{enumerate}[(a)]
            \item Observe that direct products $\Pi_{i = 1}^{n} H_i$ where $H_i \leq G_i$, are subgroups of $\Pi_{i = 1}^{n} G_i$
            \item The subsets 
            \[ \overline{G_i} = \qty{\qty(e_1, \ldots, e_{i-1}, g_i, e_{i+}, \ldots, e_n) \mid g_i \in G_i} \ \]
            where $e_i \in G_i$ is the identity element of $G_i$, are subgroups of $\Pi_{i = 1}^{n} G_i$. Moreover, it can be shown that $\overline{G_i} \cong G_i$
        \end{enumerate}
    \end{enumerate}
\end{remark}

\begin{recall}
    If $a, b \in \Z^+$, then $\text{lcm} \qty(a, b) = \frac{ab}{\text{gcd}\qty(a, b)}$
\end{recall}

\begin{theorem}
    $\Z_m \times \Z_n \cong \Z_{mn}$ if and only if $\text{gcd}\qty(m,n) = 1$
\end{theorem}

\begin{corollary}
    $\Z_{m_1} \times \Z_{m_2} \times \cdots \times \Z_{m_k} \cong \Z_{m_1m_2\cdots m_k}$ if and only if $\text{gcd} \qty(m_i, m_j) = 1$ for $i \neq j$.
\end{corollary}

\subsubsection{Subgroups Generated by a Subset}


\begin{recall} Let $G$ be a group
    \begin{enumerate}
        \item Suppose $a \in G$. Then $\langle a \rangle$ is the smallest subgroup of $G$ that contains $a$
        \item Let $\qty{H_i}_{i \in I}$ be a family of subgroups of $G$. Then $\bigcap_{i \in I} H_i$ is also a subgroup of $G$. 
    \end{enumerate}
\end{recall}

\begin{remark}
    Let $G$ be a group and let $S = \qty{a_i \in G \mid i \in I}$, where $I$ is some index set. 
    \begin{enumerate}
        \item Then the smallest subgroup of $G$ containing all of the $a_i's$ is the subgroup of $\langle S \rangle$ of $G$ generated by $S$. In particular, $\langle \qty{a} \rangle = \langle a \rangle$, for $a \in G$
        \item  If $\langle S \rangle = G$, we say that $S$ generates $G$ or $G$ is generated by $S$. We call $S$ a generating set for $G$ and the elements of $S$ are said to be generators of $G$. If $S$ is finite, then $G$ is said to be finitely generated.
        \item  Observe that the subgroup $\langle S \rangle$ is the set of all possible products, in every order, of elements of $S$ and their inverses. That is
        \[ \langle S \rangle = \qty{{a_{i_1}}^{m_1}, {a_{i_2}}^{m_2} \cdots {a_{i_n}}^{m_n} \mid n = 1, 2, \ldots, {a_i}_j \in S, m_j \in \Z }\]
        where ${a_i}_j$s are not necessarily distinct. In particular, $\langle a \rangle = \qty{a^n \in G \mid n \in \Z}$
    \end{enumerate}
\end{remark}

\begin{remark}
    If $G_1, G_2, \ldots, G_n$ are cyclic groups, then $\Pi_{i=1}^{n} G_i$ is finitely generated with generating set
    \[ \qty{\qty(a_1, e_2, \ldots, e_n), \qty(e_1, a_2, e_3, \ldots, e_n), \ldots, \qty(e_1, e_2, e_3, \ldots, e_{n-1}, a_n)} \]
    where $e_i \in G_i$ is the identity element of $G_i$ and $G_i = \langle a_i \rangle$
\end{remark}

\subsubsection{Finitely Generated Abelian Groups}

\begin{theorem}[Fundamental Theorem of Finitely Generated Abelian Group (FTFGAG) (Kronecker, 1858)] 
    Every finitely generated abelian group $G$ is isomorphic to a direct product of cyclic groups in the form
    \[ \Z_{p_1 \hspace{0.1em}^{r_1}} \times \Z_{p_2 \hspace{0.1em}^{r_2}} \times \cdots \times \Z_{p_n \hspace{0.1em}^{r_n}} \times \Z \times \Z \times \cdots \times \Z \]
    where $p_i$ are primes, not necessarily distinct, and $r_i \in \Z^+$. The direct product is unique except for possible rearrangement of factors. The number of factors $\Z$ (Betti number of $G$ or (free) rank of $G$) is unique
\end{theorem}

\begin{remark} \phantom{blank}
    \begin{enumerate}
        \item If $G$ in Theorem 1.7.3 is finite, then its rank is equal to 0
        \item To identify all abelian groups of order $n$ up to isomorphism, determine the prime factorization of $n$
    \end{enumerate}
\end{remark}

\begin{exercise}
    Enumerate all abelian groups of given order of 1350, up to isomorphism
\end{exercise}

\begin{solution}
    Note that $1350 = 2^1 \cdot 3^3 \cdot 5^2$. Thus we have 6 possible combinations listed below:
    \begin{align*}
        1350 &= 2^1 \cdot 3^3 \cdot 5^2 \\
        &= 2^1 \cdot 3^3 \cdot 5^1 \cdot 5^1 \\
        &= 2^1 \cdot 3^2 \cdot 3^1 \cdot 5^2 \\
        &= 2^1 \cdot 3^2 \cdot 3^1 \cdot 5^1 \cdot 5^1 \\
        &= 2^1 \cdot 3^1 \cdot 3^1 \cdot 3^1 \cdot 5^2 \\
        &= 2^1 \cdot 3^1 \cdot 3^1 \cdot 3^1 \cdot 5^1 \cdot 5^1
    \end{align*}
    Therefore, the possible combinations are
    \begin{align*}
        & \Z_{2} \times \Z_{27} \times \Z_{25} \\
        & \Z_{2} \times \Z_{27} \times \Z_{5} \times \Z_{5} \\
        & \Z_{2} \times \Z_{9} \times \Z_{3} \times \Z_{25} \\
        & \Z_{2} \times \Z_{9} \times \Z_{3} \times \Z_{5} \times \Z_{5} \\
        & \Z_{2} \times \Z_{3} \times \Z_{3} \times \Z_{3} \times \Z_{25} \\
        & \Z_{2} \times \Z_{3} \times \Z_{3} \times \Z_{3} \times \Z_{5} \times \Z_{5} 
    \end{align*}
\end{solution}

\begin{remark}
    Let $G$ be a finite group abelian group. If $m$ divides $\qty|G|$, then $\exists H \leq G$ such that $\qty|H| = m$
\end{remark}

\begin{exercise}
    Find a subgroup of $\Z_8 \times \Z_2 \times \Z_3 \times \Z_{25}$ of order 150
\end{exercise}

\begin{solution}
    Note that $150 = 2^1 \cdot 2^0 \cdot 3^1 \cdot 5^2$. Thus, we have a subgroup $H_2$ that is defined by
    \begin{align*}
        H_2 &= \langle 2^{3-1} \rangle \times \langle 2^{1- 0} \rangle \times \langle 3^{1 - 1} \rangle \times \langle 5^{2 - 2} \rangle \\
        &= \langle 4 \rangle \times \langle 0 \rangle \times \langle 1 \rangle \times \langle 1 \rangle \\
        &\cong \Z_2 \times \Z_3 \times \Z_{25}
    \end{align*}
\end{solution}