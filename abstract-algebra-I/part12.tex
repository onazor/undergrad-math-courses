\subsection{Friday, December 9: Ideals, Factor Rings and Ring Homomorphisms}

\subsubsection{Ideals}

\begin{definition}[Ideals]
    Let $R$ be a ring and $I$ be a subring of $R$.
    \begin{enumerate}
        \item $I$ is called a \textbf{left ideal} of $R$ if $\forall a \in I$ and $\forall r \in R, ra \in I$
        \item $I$ is called a \textbf{right ideal} of $R$ if $\forall a \in I$ and $\forall r \in R, ar \in I$
        \item $I$ is called a \textbf{(two-sided) ideal} of $R$ if $I$ is both a left and right ideal of $R$
    \end{enumerate}
\end{definition}

\begin{theorem}[Ideal Test]
    A nonempty subset $I$ of a ring $R$ is an ideal of $R$ if 
    \begin{enumerate}
        \item $\forall a, b \in I, a - b \in I$
        \item $\forall a \in I$ and $\forall r \in R, ra \in I$ and $ar \in I$
    \end{enumerate}
\end{theorem}

\begin{definition}[Principal Ideal]
    Let $R$ be a commutative ring with unity and $a \in R$. The ideal $I = \qty{ar \mid r \in R}$ of $R$, denoted by $\langle a \rangle$ is called the \textbf{principal ideal generated by} a. An ideal $I$ of $R$ is a \textbf{principal ideal} if $I = \langle a \rangle$, for some $a \in R$. 
\end{definition}

\begin{definition}[Proper Ideal]
    If $R$ is a ring then $\qty{0_R}$ (\textbf{trivial ideal}) and $R$ (\textbf{improper ideal}) are ideals. An ideal $I \neq  R$ of $R$ is referred to as a proper ideal.    
\end{definition}

\begin{exercise}
    Consider the ring $M_{2 \times 2} (\Z)$. Verify whether the following subset are ideal of $M_{2 \times 2} (\Z)$:
    \begin{enumerate}[(a)]
        \item $I = \qty{\mqty[0 & 0 \\ 0 & a] \mid a \in \Z}$
        \item $J = M_{2 \times 2} (2\Z)$
    \end{enumerate}
\end{exercise}

\begin{solution} \phantom{blank}
\begin{enumerate}[(a)]
    \item     Note that $I \subseteq M_{2 \times 2} (\Z)$ and $\mqty[0 & 0 \\ 0 & 0 ] \in I$ where $a = 0 \in \Z$. So $I \neq \varnothing$. Let $\mqty[0 & 0 \\ 0 & a ], \mqty[0 & 0 \\ 0 & b ] \in I$. Then,
    \[ \mqty[0 & 0 \\ 0 & a ] - \mqty[0 & 0 \\ 0 & b ] = \mqty[0 & 0 \\ 0 & a - b ] \in I \]
    where $a - b \in \Z$. \\
    
    Let $\mqty[x & y \\ z & w ] \in M_{2 \times 2} (\Z), \mqty[0 & 0 \\ 0 & a ] \in I$. Then, we have
    \[ \mqty[x & y \\ z & w ] \mqty[0 & 0 \\ 0 & a ] = \mqty[0 & ya \\ 0 & wa ] \notin I \text{ when 
    } ay \neq 0 \]
    and
    \[ \mqty[0 & 0 \\ 0 & a ] \mqty[x & y \\ z & w ]  = \mqty[0 & 0 \\ az & aw ] \notin I \text{ when 
    } az \neq 0 \]
    Therefore, $I$ is not an ideal of $M_{2 \times  2} (\Z)$. 
    \item Note that $\mqty[0 & 0 \\ 0 & 0 ] J$. Let $\mqty[2k_1 & 2k_2 \\ 2k_3 & 2k_4 ], \mqty[2m_1 & 2m_2 \\ 2m_3 & 2m_4 ] \in M_{2 \times 2} (2\Z)$. Then, we have
    \[\mqty[2k_1 & 2k_2 \\ 2k_3 & 2k_4 ] - \mqty[2m_1 & 2m_2 \\ 2m_3 & 2m_4 ] =  \mqty[2 (k_1 - m_1) & 2 (k_2 - m_2) \\ 2 (k_3 - m_3) & 2 (k_4 - m_4) ] \in M_{2 \times 2} (2\Z)\]
    where $k_i - m_i \in \Z \forall i = 1, 2, 3, 4$. \\
    
    Let $\mqty[x & y \\ z & w ] \in M_{2 \times 2} (\Z),  \mqty[2k_1 & 2k_2 \\ 2k_3 & 2k_4 ] \in M_{2 \times 2}(2\Z)$. Then, we have
    \begin{align*}
        \mqty[x & y \\ z & w ]  \mqty[2k_1 & 2k_2 \\ 2k_3 & 2k_4 ] &= \mqty[x2k_1 + y2k_3 & x2k_2 + y2k_4 \\ z2k_1 + w2k_3 & z2k_2 + w2k_4] \\
        &= \mqty[2(xk_1 + yk_3) & 2(xk_2 + yk_4) \\ 2(zk_1 + wk_3) & 2(zk_2 + wk_4)] \in M_{2 \times 2} (2\Z)
    \end{align*}
    where $(xk_1 + yk_3), (xk_2 + yk_4), (zk_1 + wk_3), (zk_2 + wk_4) \in \Z$. Moreover,
    \begin{align*}
        \mqty[2k_1 & 2k_2 \\ 2k_3 & 2k_4 ] \mqty[x & y \\ z & w ] &= \mqty[2k_1x + 2k_2z & 2k_1y + 2k_2w \\ 2k_3x + 2k_4z & 2k_3y + 2k_4w] \\
        &= \mqty[2(k_1x + k_2z) & 2(k_1y + k_2w) \\ 2(k_3x + k_4z) & 2(k_3y + k_4w)] \in M_{2 \times 2} (2\Z)
    \end{align*}
    where $(k_1x + k_2z), (k_1y + k_2w), (k_3x + k_4z), (k_3y + k_4w) \in \Z$. Therefore, $M_{2 \times 2} (2\Z)$ is an ideal of $M_{2 \times 2} (\Z)$.
\end{enumerate}

\end{solution}

\begin{exercise}
    Give an example to show that if $I_1$ and $I_2$ are ideals of a ring $R$, then $I_1 \cup I_2$ may not be an ideal.
\end{exercise}

\begin{solution}
    Note that $2\Z$ and $5\Z$ are ideals of $\Z$. Consider the elements $2 \in 2\Z$ and $5 \in 5\Z$. Note that $2 - 5 = -3$ but $-3 \notin 2\Z \cup 5\Z$. Therefore, it is not an ideal.
\end{solution}
\subsubsection{Factor Rings}

\begin{theorem}
    Let $R$ be a ring and $I$ an ideal of $R$. Then the collection of additive cosets $R/I$ of $I$ is a ring with binary operations
    
    \[
    \begin{cases}
        (a + I) + (b + I) &= (a + b) + I \\
        (a+I)(b+I) &= ab + I
    \end{cases}
    \]
    
    for every $a + I$, $b + I \in R/I$. 
\end{theorem}

\begin{remark} \phantom{blank}
    \begin{enumerate}
        \item We call the ring $R/I$ described in Theorem 2.5.2 the \textbf{factor ring} or \textbf{quotient ring} of $R$ modulo $I$.
        \item $a + I = b + I$ iff $a \in b + I$ iff $b \in a + I$ iff $b - a \in I$ iff $a - b \in I$
        \item Let $I$  be an ideal of a ring $R$
        \begin{enumerate}[(a)]
            \item If $R$ is a commutative ring, then so is $R/I$
            \item If $R$ has unity, then $R/I$ also has unity
        \end{enumerate}
    \end{enumerate}
\end{remark}

\begin{exercise}
    Let $I$  be an ideal of a ring $R$
    \begin{enumerate}[(a)]
        \item Prove that the associative law for multiplication and the distribute laws hold in $R/I$
        \item Prove that if $R$ is a commutative ring, then so is $R/I$
        \item Prove that if $R$ has a unity, then $R/I$ also has unity
    \end{enumerate}
\end{exercise}

\begin{solution} \phantom{blank}
    \begin{enumerate}
            \item     Let $a + I, b + 1, c + I \in R/I$ where $a, b, c \in R$. Then,
                \begin{align*}
                    \qty(a + I) \qty[(b + I)( c + I)] &= (a + I)(bc + I) \\
                    &= abc + I \\
                    &= (ab + I)(c + I) \\
                    &= \qty[(a + I)(b + I)](c + I)
                \end{align*}
                and for LDL and RDL,
                \begin{align*}
                    (a + I)\qty[(b + I) + (c + I)] &= (a + I)(b + c + I) \\
                    &= a(b + c) + I \\
                    &= ab + ac + I \\
                    &= ab + I + ac + I \\
                    &= (a + I)(b + I) + (a + I)(c + I)
                \end{align*}
                and 
                \begin{align*}
                    \qty[(a + I) + (b + I)](c + I) &= (a + b + I)(c + I) \\
                    &= (a + b)c + I \\
                    &= ac + bc + I \\
                    &= ac + I + bc + I \\
                    &= (a + I)(c + I) + (b + I)(c + I)
                \end{align*}
            \item Suppose $R$ is a commutative ring. Let $a + I$, $b + I \in R/I$. Then,
            \[ (a + I)(b + I) =  ab + I = ba + I = (b + I)(a + I)\]
            Therefore, $R/I$ is also a commutative ring.
            \item Suppose $1_R \in R$. Let $a + I \in R/I$. Then,
            \[ (a + I)(1_R + I) = a \cdot 1_R + I = a + I = 1_R \cdot a + I = (1_R + I)(a + I) \]
            Therefore, $1_R + I$ is the identity element under multiplication. By Cayley tables,
            
            \begin{center}
                \begin{multicols}{2}
                
                    \begin{tabular}{c | c c c}
                        $+$ & $6\Z$ & $2 + 6\Z$ & $4 + 6\Z$ \\
                        \hline
                        $6\Z$ & $6\Z$ & $2 + 6\Z$ & $4 + 6\Z$ \\
                        $2 + 6\Z$ & $2 + 6\Z$ & $4 + 6\Z$ & $6\Z$ \\
                        $4 + 6\Z$ & $4 + 6\Z$ & $6\Z$ & $2 + 6\Z$ \\
                    \end{tabular}
                    
                    \columnbreak
                    \hspace{2em}
                    \begin{tabular}{c | c c c}
                        $\cdot$ & $6\Z$ & $2 + 6\Z$ & $4 + 6\Z$ \\
                        \hline
                        $6\Z$ & $6\Z$ & $6\Z$ & $6\Z$ \\
                        $2 + 6\Z$ & $6\Z$ & $4 + 6\Z$ & $2 + 6\Z$ \\
                        $4 + 6\Z$ & $6\Z$ & $2 + 6\Z$ & $4 + 6\Z$ \\
                    \end{tabular}
                \end{multicols}
        \end{center}
        
        Since $2\Z$ is a commutative ring, $2\Z / 6\Z$ is also a commutative ring. Moreover, the Cayley tables shows that $1_{2\Z / 6\Z} = 4 + 6\Z$ and $(2 + 6\Z)^{-1} = 2 + 6\Z$ and $(4 + 6\Z)^{-1} = 4 + 6\Z$. Therefore, $2\Z / 6\Z$ is a field. \qedsymbol
    \end{enumerate}
\end{solution}

\begin{exercise}
    Show that $6\Z$ is an ideal of $2\Z$ and $2\Z / 6\Z$ is a field.
\end{exercise}

\begin{proof}
    Note that $6\Z \subseteq 2\Z$ and $0 = 6 \cdot 0 \in 6\Z \neq \varnothing$. Let $6m, 6k \in 6\Z$ ($m, k \in \Z$). Then,
    \[ 6a - 6b = 6(a - b) \in 6\Z \text{ where } a- b \in \Z \]
    Let $2n \in 2\Z$, $6m \in 6\Z$ ($n, m \in \Z$). Then,
    \[2n \cdot 6m = 6m \cdot 2n = 6(2mn) \in 6\Z \]
    where $2mn \in \Z$. Therefore, $6\Z$ is an ideal of $2\Z$. We need to show that $2\Z / 6\Z$ is a field. \\
    
    Since $6\Z$ is an ideal of $2\Z$, $2\Z / 6\Z$ is a ring with the following addition and multiplication tables.
\end{proof}

\subsubsection{Ring Homomorphisms}

\begin{definition}
    A \textbf{ring homomorphism} of a ring $\langle R, +, \cdot \rangle$ into a ring $\langle R', +', \cdot' \rangle$ is a function $f : R \to R'$ such that $\forall a, b \in R$,
    \[ f(a + b) = f(a) +' f(b) \]
    and 
    \[ f(a \cdot b) = f(a) \cdot' f(b) \]
\end{definition}

\begin{remark}
    Let $f : R \to R'$  be a ring homomorphism
    \begin{enumerate}
        \item If $f$ is onto, then $f$ is called a \textbf{ring epimorphism}
        \item If $f$ is one-to-one, then $f$ is called a \textbf{ring monomorphism}
        \item If $f$ is bijective, then $f$ is called a \textbf{ring isomorphism}. If $R = R'$, we also call $f$ a \textbf{ring automorphism}
    \end{enumerate}
\end{remark}

\begin{definition}[Isomorphic Rings]
    Two rings $R$ and $R'$ are said to be \textbf{isomorphic} written $R \cong R'$ if there exists a (ring) isomorphism of a ring $R$ with a ring $R'$.
\end{definition}

\begin{remark} \phantom{blank}
    \begin{enumerate}
        \item  Isomorphism between rings define an equivalence relation on any collection of rings.
        \item If $f : R \to R'$ is a ring homomorphism, then $f : \langle R, + \rangle \to \langle R', +' \rangle$ is a group homomorphism. Hence, all results from group homomorphism still hold:
            \begin{enumerate}[(a)]
                \item $f(0_R) = 0_{R'}$
                \item If $a \in R$, then $f(-a) = -f(a)$
                \item For any $m \in \Z$, $f(ma) = mf(a)$
            \end{enumerate}
    \end{enumerate}
\end{remark}

\begin{exercise}
    Show that $f: \C \to M_{2 \times 2} (\R)$ given by 
    \[ f(a + bi) = \mqty[a & b \\ -b & a] \]
    for $a, b \in \R$ is a ring homomorphism
\end{exercise}

\begin{proof}
 Let $a + bi, c + di \in \C$. Then,
    \begin{align*}
        f((a + bi) + (c + di)) &= f(a + c + (b + d)i) \\
        &= \mqty[a + c & b + d \\ - (b + d) & a + c]  \\
        &= \mqty[a + c & b + d \\ - b - d & a + c]  \\
        &= \mqty[a & b \\ - b & a]  + \mqty[c & d \\ - d & c] \\
        &= f(a + bi) + f(c + di)
    \end{align*}
    and 
    \begin{align*}
        f((a + bi)(c + di)) &= f((ac - bd) + (ad + bc)i) \\
        &= \mqty[ac - bd & ad + bc \\ - (ad + bc) & ac - bd] \\
        &= \mqty[ac - bd & ad + bc \\ - ad - bc & ac - bd] \\
        &= \mqty[a & b \\ -b & a]\mqty[c & d \\ -d & c] \\
        &= f(a + bi) f(c + di)
    \end{align*}
    Therefore $f$ is a ring homomorphism. \qedsymbol
\end{proof}

\begin{exercise}
    Determine all ring homomorphisms from $\Z_4 \to \Z_{12}$
\end{exercise}

\begin{proof}
    Suppose $\phi: \Z_4 \to \Z_{12}$ is a ring homomorphism. Then, we have
    \begin{align*}
        \phi :  \Z_4 &\to \Z_{12} \\
        g &\mapsto  g \phi(1)
    \end{align*}
    For the group homomorphisms, note that $\qty|\phi(1)| = \qty|1| = 4 \Rightarrow \qty|\phi(1)| = 1, 2, 4$ and $\qty|\phi(1)| \mid \qty|\Z_{12}| = 12 \Rightarrow \qty|\phi(1)| = 1, 2, 3, 4, 6, 12$. So, $\qty|\phi(1)| = 1, 2, 4$. Therefore, $\phi(1) = 0, 6, 3, 9$. \\
    
    For the ring homomorphism, to preserve multiplication, note that $1 \cdot 1 = 1 \in \Z_4$, so 
    \[ \phi(1 \cdot 1) = \phi(1) \phi(1) = \phi(1) \Rightarrow \qty(\phi(1))^2 = \phi(1) \]
    Note that $0^2 = 0, 6^2 = 0 \neq 6, 3^2 = 9 \neq 3, 9^2 = 9$. Hence $\phi(1) = 0, 9$. \qedsymbol
\end{proof}

\begin{theorem} Let $f: R \to R'$  be a ring homomorphism

    \begin{enumerate}
        \item If $S$ is a subring of $R$, then $f(S) = \qty{f(a) \mid a \in S}$ is a subring of $R'$
        \item If $S'$ is a subring of $R'$, then $f^{-1}(S') = \qty{a \in R \mid  f(a) \in S'}$ is a subring of $R$
        \item If $R$ is a commutative ring, then $f(R)$ is also a commutative ring
        \item Let $R$  be a ring with unity $1_R$ and $R' \neq \qty{0_{R'}}$
        \begin{enumerate}[(a)]
            \item Then $f(R)$ has unity $f(1_R)$
            \item If $a \in R$ is a  unit, then $f(a)$ is a unit in $f(R)$ with $(f(a))^{-1} = f(a^{-1})$
            \item If $a \in R$ and $n \in \Z^{+}$, then $f(a^n) = (f(a))^n$
        \end{enumerate}
    \end{enumerate}
\end{theorem}

\begin{definition}
    Let $f: R \to R'$  be a ring homomorphism. The \textbf{Kernel} of $f$ is the set
    \[ \text{Ker}f = \qty{x \in R \mid f(x) = 0_{R'}} = f^{-1}(\qty{0_{R'}}) \]
\end{definition}

\begin{remark}
    Let $f: R \to R'$  be a ring homomorphism.
    \begin{enumerate}
        \item Then $f$ is an isomorphism if and only if $f$ is onto and Ker$f = \qty{0_R}$
        \item If $g \in R, g' \in R'$ and $f(g) = g'$, then 
        \[ f^{-1}(\qty{g'}) = \qty{x \in R \mid f(x) = g'} = g + \text{Ker}f \]
        \item If $I$ is an ideal of $R$, then $f(I)$ is an ideal of $f(R)$
        \item If $I'$ is an ideal of $R'$, then $f^{-1}(I')$ is an ideal of $R$
    \end{enumerate}
\end{remark}

\begin{theorem}
    Let $f : R \to R'$ be a ring homomorphism. Then $\text{Ker}f$ is an ideal of $R$
\end{theorem}

\begin{theorem}
    Let $I$ be an ideal in a ring $R$. Then the map $\pi : R \to R/I$ given by $\pi(r) = r + I$ is a ring epimorphism with Ker$\pi = I$.
\end{theorem}

\begin{remark}
    The mapping $\pi$ is called the \textbf{natural homomorphism} from $R \to R/I$
\end{remark}

\begin{theorem}[First Isomorphism Theorem for Rings]
    Let $f : R \to R'$  be a ring homomorphism. Then $R/\text{Ker}f \cong f(R)$
\end{theorem}